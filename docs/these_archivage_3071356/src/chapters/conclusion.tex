\chaptertitle{Conclusions et perspectives}
\label{chap:conclusion}
\introformatting

Dans cette dernière partie, nous ferons le bilan de nos contributions puis nous
détaillerons les perspectives que nous avons identifiées à partir des travaux
que nous avons menés.

\section{Contributions}
\label{sec:conclusion-contribs}

L'une des préoccupations principales tout au long de ce travail de thèse a été
de tenter de s'abstraire d'un grand nombre d'hypothèses implicites du domaine,
pour les réexaminer à l'aune d'une approche qui se voulait à la fois rigoureuse
et pragmatique. Notre première contribution a donc été d'adopter une posture
volontairement agnostique et de {\bf reprendre l'interprétation et la
formalisation de nombreux objets}. L'interprétation rigoureuse de \traceroute
(\refchap{traceroute}), le fonctionnement d'\udpping (\refchap{udpping}), et
l'interprétation des tables de transmission (\refchap{revtables}), résultent de
cette approche sciemment naïve. Notre contribution se situe notamment dans une
description pragmatique qui tient compte de l'interprétation assez libérale des
RFCs et autres documents normatifs par les acteurs réels. Notre interprétation
nous a permis de mieux comprendre les limites de certaines approches
historiques, mais également d'ouvrir la voie pour de nouvelles approches.

C'est ainsi que nous avons décidé d'utiliser \traceroute non pas pour collecter
des cartes, mais pour tenter de {\bf mesurer directement une propriété
topologique qui nous intéressait} : la distribution de degré. Dans ce contexte,
nous avons dû remettre en question la définition même de distribution de degré,
et distinguer différentes modélisation du même objet, en l'occurence le réseau
\LL et le réseau \LLL. La méthode de mesure que nous avons établi avec
\traceroute nous permet d'{\bf obtenir la liste d'au moins une adresse de chacun
des voisins au niveau \LLL d'un routeur cible}, pour un coût total (en terme de
charge réseau) très limité, une approche validée par la simulation.
Toutefois, cette approche suppose que le voisinnage de la cible ne bloque pas
les paquets \icmp, une hypothèse qui tend à être remise en cause.

Cette limitation nous a incité à réaliser une analyse de l'échantillonage des
degrés dans un graphe, et nous a permis de conclure qu'il suffisait pour estimer
correctement la distribution de degré d'un échantillon de taille relativement
limitée (quelques milliers de n\oe{}uds), pour peu qu'on dispose d'une
estimation très précise du degré de chaque n\oe{}ud. C'est ainsi que nous avons
conçu la primitive \udpping (\refchap{udpping}) : une {\bf primitive de mesure
très exigeante sur la cible, mais qui estime très précisément son degré}.
Afin d'en tirer l'estimation de la propriété qui nous intéressait, nous avons
également réalisé un {\bf protocole de mesure}, capable non seulement d'obtenir
la propriété recherchée (le nombre d'interfaces de la cible), mais également de
détecter les cas pouvant biaiser la distribution finale. Nous avons
notamment implémenté un outil dérivé d'\udpping, \udpexplore, qui permet
d'identifier la liste des hôtes séparant un moniteur du c\oe{}ur de la topologie
physique. Un des aspects les plus importants de notre contribution porte
précisément sur une analyse rigoureuse des effets de certaines configurations
topologiques sur notre primitive de mesure, comment les détecter, et comment les
corriger. Cette analyse est validée par des simulations sur plusieurs modèles de
topologies, en particulier des topologies homogènes et hétérogènes. Une de nos
conclusions est que nous sommes capables d'{\bf estimer avec une très grande
confiance les fractions de n\oe{}uds de faible degré (<10)} dans le c\oe{}ur de
la topologie physique, et qu'à l'inverse, nous sommes capables de {\bf majorer
avec une grande confiance la fraction des n\oe{}uds de degré supérieur à un
certain degré $d$}, en extrapolant leur fréquence d'apparition théorique à
partir de leur fréquence d'apparition dans notre échantillon. Cette dernière
contribution est une contribution importante, notamment puisqu'elle repose sur
un raisonnement solide et des hypothèses très clairement établies.

En complément de l'objet que nous souhaitions initialement mesurer, la
distribution de degré, notre primitive \udpping nous a permis d'obtenir une
information plus précise que le nombre de leurs interfaces pour les cibles qui y
répondent. En effet, dans une certaine mesure, nous avons établi une manière
d'{\bf estimer la table de transmission} (\refchap{revtables}) d'un routeur qui
répond aux sondes \udpping. Si cette information n'est pas nécessairement simple
à exploiter avec des modèles de graphes très généralistes, elle donne
indiscutablement une information beaucoup plus riche que le simple degré sur la
manière dont l'information {\em circule} à travers le réseau.

Outre les résultats que nous avons obtenu sur la distribution de degré aux
niveaux \LL et \LLL, et les tables de transmission, cette thèse a été l'occasion
d'aborder la problématique plus générale de la métrologie des grands réseaux.
Nous avons esquissé, à travers la mesure d'Internet, la problématique de {\bf
l'estimation d'une propriété topologique d'un grand réseau réel à l'aide d'une
primitive de mesure}. La méthodologie que nous avons employée ici puise
largement dans une connaissance du domaine d'application spécifique du réseau
Internet, mais elle peut être conceptuellement étendue à beaucoup d'autres
cadres; c'est d'ailleurs l'une de nos principales perspectives
(\refsubsec{conclusion-sampling}). Plus généralement, nous avons effectué un
travail exploratoire, qui, au-delà des résultats direct, a levé le voile sur un
certain nombre de perspectives. La {\bf description et l'organisation de ces
perspectives} présentée ci-dessous constitue une contribution propre.

\section{Perspectives}
\label{sec:conclusion-perspectives}

Nous avons organisé les perspectives suscitées par nos travaux en trois
catégories : celles qui relèvent de l'approfondissement de l'utilisation
d'\udpping (\refsubsec{conclusion-udpping}), celles qui généralisent l'approche
d'échantillonage orienté propriété (\refsubsec{conclusion-sampling}) à d'autres
propriétés ou d'autres réseaux, et enfin, des nouveaux objets de la topologie
d'Internet (\refsubsec{conclusion-objects}) qui nous semblent être d'un intérêt
particulier.

\subsection{Approfondissement d'\udpping}
\label{subsec:conclusion-udpping}

L'outil \udpping que nous avons mis au point et dont nous avons analysé le
fonctionnement représente une primitive de mesure précise et fiable, que nous
avons conçu pour identifier les interfaces des routeurs du c\oe{}ur
(\refchap{udpping}) et qualifier leur utilisation (\refchap{revtables}).
Cependant, un certain nombre de questions restent ouvertes et pourraient faire
l'objet d'approfondissements.

\subsubsection{Validation de la mesure}

L'une des perspectives les plus importantes pour la validation de notre travail,
outre les garanties théoriques et simulatoires que nous avons déjà présenté,
serait de confronter les interfaces identifiées par \udpping avec la liste des
interfaces réelles d'un routeur du c\oe{}ur.

La mesure réalisée par \udpping est d'autant plus difficile à valider qu'il
n'existe bien sûr pas d'annuaire public des interfaces des routeurs du c\oe{}ur
(sans quoi la mesure serait d'un intérêt très limité). Pour valider la liste des
interfaces obtenues avec \udpping distribué, il faudrait disposer d'un accès à
la liste des interfaces d'un routeur du c\oe{}ur qui répond à \udpping. Or, les
routeurs du c\oe{}ur appartiennent le plus souvent à des institutions de grande
ampleur, qui ne sont pas nécessairement enclines à confier un accès de cette
nature.
Les routeurs du c\oe{}ur sont en effet souvent des machines extrêmement onéreuses
(les prix se chiffrent au minimum en dizaines de milliers de dollars, allant
jusqu'à plusieurs centaines de milliers) et leur configuration peut être
considérée comme une information stratégique. Valider le fonctionnement
d'\udpping distribué de cette manière suppose donc un travail délicat, afin
de convaincre l'autorité responsable d'un tel routeur de fournir la liste de ses
interfaces et, idéalement, sa table de transmission. Cette perspective nous
semble toutefois importante dans la mesure où elle permettrait de valider ou
d'invalider clairement les résultats de notre mesure et notre méthode.

\subsubsection{Dynamique et mesure longue}

Notre travail avec \udpping avait pour objectif de capturer une estimation de la
distribution de degré des routeurs du c\oe{}ur de la topologie physique à un
instant donné, et à cet effet, nous avons paramétré notre mesure afin qu'elle
soit la plus brève possible (sans surcharger les cibles) pour minimiser l'effet
de la dynamique du réseau.

Pourtant, puisque nous avons justifié la validité de notre protocole de mesure
"instantané", on peut précisément envisager de répéter cette mesure
régulièrement au cours du temps, pour capturer la dynamique à moyen et long
terme. Une hypothèse fréquente est que le c\oe{}ur d'Internet est relativement
statique, mais cette hypothèse n'est pas étayée par des mesures fiables. Nous
proposons deux protocoles complémentaires afin d'explorer cette piste. Le
premier consiste à se fixer un ensemble de cibles répondant à \udpping et de
répéter la mesure \udpping distribué à intervalle régulier pendant une longue
durée, par exemple toutes les 24h pendant 6 mois, et d'observer l'évolution de
la liste des interfaces de chaque cible pendant cette période. Cette mesure
permettrait de mesurer l'évolution de la liste des interfaces de {\em chaque}
routeur de notre ensemble de cibles. Le second protocole, complémentaire du
premier, consiste à s'intéresser non pas à un ensemble de cibles fixé, mais à
évaluer à la même fréquence la distribution de degré des routeur du c\oe{}ur de
la topologie physique, c'est à dire à réaliser un nouvel échantillonage uniforme
à chaque itération. De cette manière, on bénéficie du travail d'analyse que nous
avons décrit au \refchap{udpping}, et on peut s'intéresser à l'évolution de la
{\em distribution} plutôt qu'à l'évolution des listes d'interfaces, plus
exposées aux artefacts de mesure. Ce second protocole est cependant plus coûteux
pour le réseau, puisque la construction d'une liste uniforme de cibles requiert
l'envoi d'un très grand nombre de sondes \udpping à la recherche de cibles y
répondant.

\subsubsection{Autres ensembles de moniteurs}

L'une des problématiques mises en évidence par tous nos travaux sur la mesure
de la topologie d'Internet est l'importance de l'ensemble des moniteurs, ou
points d'observation, depuis lesquels on mesure la topologie. C'est l'une des
principales faiblesses de la méthode historique, et c'est ce constat qui en
premier lieu nous a servi de piste pour mettre au point nos méthodes de mesure
distribuées.

Pour réaliser nos mesures, nous avons utilisé à plusieurs reprises
l'infrastructure PlanetLab. C'est une infrastructure riche et très libérale dans
ses conditions d'utilisation, qui nous a permis d'établir la faisabilité et la
pertinence de notre approche. Cependant, elle reste limitée, à la fois en termes
de nombre de moniteurs (quelques centaines), et surtout en termes de diversité
des sous-réseaux. En effet, ils n'appartiennent qu'à une collection relativement
limitée d'AS (environ 200 au moment de nos mesures) et pour beaucoup sont
hébergés dans des réseaux académiques. En conséquence, on peut craindre qu'ils
ne fournissent qu'une vision partielle du réseau. S'il est peu vraissemblable
que cela impacte l'estimation de la fraction des n\oe{}uds de degrés faibles, en
revanche, il est clair que cela peut avoir un effet sur l'estimation de la
fraction des n\oe{}uds de degré fort. Nos travaux sur les tables de tranmission
suggèrent également que dans un tel cadre, l'ensemble des moniteurs de PlanetLab
est insuffisant pour obtenir une vision fiable de l'objet mesuré.

Les prérequis pour effectuer une mesure \udpping depuis une machine hôte sont
assez simples, mais exigeants: il faut pouvoir émettre des paquets \udp
arbitraires, et surtout disposer d'un accès non filtré au traffic \icmp entrant
pour pouvoir écouter les réponses. Le premier critère est assez facile à
obtenir, mais le second exige souvent des privilèges particuliers sur un réseau
qui ne filtre pas le traffic \icmp. Nous avons envisagé plusieurs options, dont
certaines sont plus ou moins faciles à mettre en oeuvre :
\begin{itemize}
  \item {\em DIMES}~\cite{dimes}: Le projet DIMES rassemble des
  participants volontaires exécutant sur leur machine, souvent personelle, un programme agent qui permet
  d'exécuter des mesures. Il est aujourd'hui surtout utilisé pour collecter des
  résultats de \traceroute, mais pourrait être adapté pour réaliser des mesures
  \udpping.
  \item {\em Looking glasses}, {\em routeviews}~\cite{routeviews}: Mis à la
  disposition du public principalement afin de partager des informations
  relatives aux protocoles de routages par les fournisseurs d'accès à Internet
  ou les points d'échange Internet, les serveurs de {\em looking glasses} fournissent en général la possibilité de consulter
  un annuaire \bgp, ou d'exécuter \icmp~\ping, \tcp~\ping ou \traceroute.
  Déployer \udpping sur ces serveurs requiert l'accord de parties privées ou
  industrielles, mais offrirait un point de vue très riche, au sein du c\oe{}ur
  d'Internet.
  \item {\em RIPE Atlas}~\cite{ripeatlas}: {\em RIPE Atlas} déploie
  gratuitement, auprès d'institutions volontaires, un serveur de mesures embarqué, constitué d'une
  puce alimentée par USB et connectée à Internet via un port RJ45 traditionnel.
  Sa puce contient un système d'exploitation simplifié dédié à effectuer des
  mesures telles que \ping ou \traceroute. À mesure de son déploiement, le
  réseau RIPE Atlas pourrait devenir un complément intéressant à
  PlanetLab.
  \item {\em CAIDA Ark}~\cite{caida}: L'infrastructure {\em Ark} utilisée par
  {\em CAIDA} pour réaliser ses prélèvements \traceroute distribués est
  semblable à celle de {\em RIPE Atlas} et de {\em PlanetLab}: des institutions
  volontaires hébergent des machines dédiées à réaliser des prélèvements
  distribués et comprend plusieurs dizaines de moniteurs répartis dans le monde.
  Une version embarquée basée sur le {\em Raspberry Pi} est très proche des sondes {\em RIPE Atlas}. Certains moniteurs {\em Ark} ont déjà été
  utilisés pour des mesures tierces, comme par exemple {\em
  MERLIN}~\cite{merlin}. Cette infrastructure est déjà coutumière des mesures
  distribuées à très grande échelle, et serait donc un complément naturel à
  PlanetLab, bien que ses moniteurs soient vraissemblablement au moins en partie
  redondants.
  \item Hébergeurs commerciaux: Afin de proposer des CDNs\footnote{{\em
  Content Delivery Networks}, réseaux de diffusion de contenu} efficaces, de
  nombreux hébergeurs commerciaux, tels qu'Amazon Web Services~\cite{aws} ou
  Microsoft Azure~\cite{msazure}, ont déployé des serveurs localisés dans le
  monde entier, et offrent des machines virtuelles ou dédiées à leurs clients.
  Bien que le nombre de sites soit très limité (quelques dizaines), ils sont
  choisis par leurs propriétaires précisément pour leur bonne répartition dans
  le réseau, et pourraient donc offrir une vision de la topologie complémentaire
  à celle fournie par les hôtes académiques.
  \item Extension de navigateur ou applet: Les navigateurs Internet ne
  fournissent pas d'API exposant les fonctionalités de réseau bas niveau
  (particulièrement pas l'écoute d'\icmp), mais en revanche, les extensions de
  navigateur ou les applets (Java, Flash, Silverlight\ldots) peuvent, avec
  l'autorisation du client, exécuter du code relativement arbitraire. Développer
  un agent de mesure sous cette forme, pour \udpping comme pour d'autres
  primitives, pourrait être un complément crédible à un projet tel que {\em
  DIMES}. Son intérêt résiderait dans la simplicité pour l'utilisateur de
  participer à des mesures, mais il pose la question de la faisabilité de son
  déploiement à grande échelle.
  \item Application mobile : Les appareils connectés mobiles, tels que
  les téléphones ou les tablettes, exposent le plus souvent aux applications un
  accès, même limité, à la couche réseau. Ces terminaux présentent un intérêt en
  termes de diversité des localisations et des routes, mais également un certain
  nombre de difficultés. Sur les systèmes d'exploitations les plus utilisés du
  marché, l'accès aux {\em sockets} privilégiés n'est pas disponible par défaut.
  Certaines technologies de connectivité sans fil passent par des réseaux
  virtuels (opaques aux niveau \LL et \LLL), ce qui complexifie l'interprétation
  des mesures. Enfin, ces appareils ont souvent des capacités en énergie
  limitées, et une activité réseau prolongée peut induire une forte
  consommation.
  \item Sondes embarquées : À la lisière entre l'application mobile et les
  sondes de {\em RIPE Atlas}, les {\em dongles} ARM\footnote{À titre d'exemple,
  la clé multimedia {\em Google Chromecast}~\cite{chromecast} repose sur un
  {\em dongle} ARM de ce type.} semblent être une opportunité très prometteuse.
  Fabriqués à très bas coût (de l'ordre entre 5€ et 20€ à l'unité), ces
  terminaux de quelques centimètres carrés alimentés en USB embarquent des architectures ARM largement assez
  performantes pour réaliser des mesures réseau, et même effectuer une partie
  de leur traitement localement. Plusieurs systèmes d'exploitations très
  documentés supportent ce type de terminaux et rendent trivial le portage des
  primitives de mesure. Tout comme pour les sondes {\em RIPE Atlas}, il se pose
  la question du déploiement et de la maintenance occasionelle.
\end{itemize}

Pour toutes ces solutions, un problème commun demeure : la difficulté pour un
réseau ciblé (ou simplement traversé) par les sondes \udpping à différencier une
activité de mesure d'une activité hostile de type déni de service. En effet, les
administrateurs de réseau sont souvent peu enclins à autoriser les hôtes à
émettre des paquets \udp vers des destinations aléatoires, car ils s'exposent
alors au risque du {\em blacklisting}. Cependant, de notre expérience pratique,
seuls des réseaux mal configurés (principalement au niveau des tables \udp des
pare-feux) sont susceptibles de tomber en panne à cause d'une mesure \udpping
distribuée.

\subsubsection{Approche complémentaire pour la bordure}

La méthode de mesure du degré par des moniteurs distribués vers des cibles
aléatoires est conçue pour mesurer le degré des n\oe{}uds du c\oe{}ur d'Internet.
En effet, comme détaillé au \refchap{traceroute} et au \refchap{udpping}, les
n\oe{}uds de la bordure d'Internet ont des interfaces intrinsèquement difficiles à
observer à l'aide d'une mesure distribuée telle que celle que nous emloyons.

Nous pensons que cette limitation n'a que peu d'impact conceptuel, dans la
mesure où la complexité du routage et la plupart des problématiques associées se
situent dans le c\oe{}ur d'Internet, mais afin d'établir la distribution de degré
de la topologie complète et pas simplement du c\oe{}ur, il faudrait être
capable de mesurer la distribution de degré de la bordure d'Internet. Une
approche complémentaire doit donc être menée, attachée spécifiquement à mesurer
cette propriété. Comme par définition la bordure d'Internet est constituée
d'arbres, une piste pourrait être de réaliser certaines hypothèses sur ces
arbres (par exemple sur leur régularité), et de tenter de mesurer les paramètres
typiques de ces modèles dans la réalité (par exemple la profondeur et le degré
moyen de ces arbres).

\subsection{Echantillonage orienté propriété}
\label{subsec:conclusion-sampling}

Notre approche est conceptuellement novatrice dans ce qu'elle s'attaque à
mesurer directement une propriété topologique sur un réseau réel plutôt qu'à
tenter d'établir une carte sur laquelle on lirait ensuite cette propriété. Plus
précisément, l'approche est de bâtir une primitive de mesure qui permet
d'estimer une certaine propriété $p(u)$ d'un n\oe{}ud (ou d'une arete) $u$ d'un
sous-ensemble $V$ de n\oe{}uds (ou d'aretes) du réseau $G$, et une méthode qui
permet d'échantilloner des n\oe{}uds (ou des aretes) dans ce sous-ensemble,
d'une manière qui soit uniforme vis à vis de la propriété mesurée, ou au moins
dont la non-uniformité est maîtrisée (c'est à dire qu'on peut la corriger
\aposteriori). Dans le cas d'\udpping distribué, par exemple, la propriété
estimée est le degré (nombre d'interfaces), le réseau $G$ est le c\oe{}ur
physique d'Internet, et $V$ est le sous-ensemble des n\oe{}uds répondant à
\udpping.

Nous pensons que cette approche simple peut être déclinée, à la fois à d'autres
propriétés ou d'autres réseaux, et avec d'autres méthodes d'échantillonage.

\subsubsection{Application à d'autres réseaux}

De nombreux réseaux, à l'instar d'Internet, reposent sur une indexation
à 32 bits ou moins, et à ce titre, sont aisés à échantilloner de manière
uniforme.
C'était par exemple le cas des utilisateurs Facebook jusqu'à 2007, des
commentaires sur le site Slashdot, ou encore des utilisateurs du jeu League of
Legends sur une région donnée. La plupart de ces réseaux offrent une API
publique \http, mais dont l'accès est souvent strictement limité en nombre de
requêtes par unité de temps. Leur cartographie complète est donc peu
vraissemblable, mais un échantillonage adapté peut permettre d'estimer des
propriétés topologiques des réseaux sous-jacents, pour peu que l'on démontre
l'absence de biais indirect lié à l'attribution d'un identifiant (par exemple,
lié à l'ancienneté d'un compte).

\subsubsection{Marche aléatoire orientée propriété}

Hormis l'échantillonage uniforme lié à l'indexation numérique, une autre méthode
d'échantillonage intéressante et plus spécifique aux graphes est la marche
aléatoire. La difficulté repose alors sur l'uniformité de l'échantillonage
relativement à la propriété mesurée, mais une étude rigoureuse peut souvent
permettre de s'en abstraire. Cette approche a par exemple été mise en oeuvre
dans le cas de l'échantillonage sans biais relatif au degré dans le cas de
Facebook~\cite{gjoka2010walking}. Des techniques similaires peuvent également
s'appliquer à des réseaux qu'il est naturel de parcourir avec des marches
aléatoires, comme par exemple les réseaux de pages web ou les réseaux
P2P\footnote{{\em Peer to peer}, pair-à-pair.}.

\subsection{Nouveaux objets d'intérêt}
\label{subsec:conclusion-objects}
\subsubsection{Réseau de routage pondéré}

Au moment de valider notre première mesure \traceroute distribuée, et à nouveau
pour valider la mesure \udpping distribuée, nous avons été confrontés à la même
interrogation : que penser d'une interface d'une cible qui n'est observée que
par un seul ou un très petit nombre de moniteurs ? Cette interface est-elle un
faux positif, c'est à dire une interface n'appartenant pas réellement à la cible
mais observée par erreur par un petit nombre de moniteurs, ou une interface
"rare", difficile à observer ? Dans ce dernier cas, comment justifier le peu de
robustesse de la notion de degré dans un cas où manifestement, on aurait
aisément pu "rater" cette interface en supprimant un seul moniteur ?

La question plus générale qui se pose est celle de l'importance relative des
interfaces d'un routeur (ou des arêtes d'un n\oe{}ud) dans leur activité de
routage. On peut concevoir cette idée soit \apriori, comme une {\em probabilité
d'utilisation} ou une table de routage, soit \aposteriori, comme un {\em taux
d'utilisation}. Dans les deux cas il s'agit de considérer la matrice d'adjacence
de la topologie physique, pondérée par des réels qui représentent l'importance
d'une interface dans son activité de routage. Cette notion, qui nous semble être
le réel objet d'importance derrière la distribution de degré pour l'étude du
routage, qu'il soit en modélisation ou en analyse, a inspiré nos travaux sur les
tables de tranmission, mais il reste à compléter, notamment dans son approche
formelle et son lien avec les algorithmes de routage. Une estimation de cette
matrice sur le réseau réel d'une part, sur des graphes synthétiques d'autre
part, nous permettrait de valider ou d'invalider des modèles de topologies.

\subsubsection{Topologie égo-centrée avec \udpexplore}

Bien que les sondes d'\udpping soient à la base conçue afin de lister toutes les
interfaces dans le c\oe{}ur d'une cible donnée à partir d'un ensemble distribué
de moniteurs, nous avons montré que nous pouvions utiliser des sondes similaires
sous la forme de l'outil \udpexplore. \udpexplore est utilisé dans notre travail
comme un simple outil de validation pour détecter le cas où deux moniteurs sont
dans le même sous-arbre de la bordure, mais il permet de lister depuis un
moniteur d'une part toutes les interfaces tournées vers le c\oe{}ur qui
appartiennent à des n\oe{}uds soit dans le c\oe{}ur, soit dans d'autres
sous-arbres, d'autre part toutes les interfaces tournées vers le moniteur de
n\oe{}uds qui sont plus haut dans le même sous-arbre.

Plus généralement, l'objet auquel donne accès les sondes \udpping depuis un
moniteur est l'ensemble des interfaces à une certaine distance $d$ en termes de
{\em hops} au niveau \ip d'un n\oe{}ud et qui sont soit des interfaces tournées
vers le moniteur de n\oe{}uds dans le même sous arbre que lui tournées vers le
c\oe{}ur, soit des interfaces tournées vers le c\oe{}ur de n\oe{}uds qui ne sont
pas dans le même sous-arbre.
En plus d'un intérêt propre à cet objet (qui donne en quelque sorte la vision
projetée d'un n\oe{}ud sur la topologie du c\oe{}ur), c'est un critère de plus
pour valider ou invalider un modèle de topologie synthétique, puisqu'il s'agit
d'une propriété qu'on peut également calculer sur un réseau formel ou un réseau
simulé et la confronter aux observations. Cette approche peut compléter celle
qui a déjà été menée par Latapy \etal~\cite{latapy2008radar}.

\subsubsection{Routes longues}

Lors de nos travaux préliminaires sur \traceroute, nous avons été amenés à
réaliser des études sur la longueur des routes sous-jacentes à nos observations.
Plus précisément, nous avons constaté que si nous répétions \traceroute depuis
un moniteur donné vers une cible donnée, alors le \ttl nécessaire avant
d'atteindre la cible pouvait montrer une variabilité.

Or, \traceroute s'arrête généralement d'envoyer des sondes dès qu'une sonde
\icmp répond. Des travaux ultérieurs ont montré que des sondes \icmp successives
pouvaient emprunter des routes différentes, et qu'une sonde de \traceroute peut
être interprété comme un certain tirage aléatoire d'une des routes parmi les
routes possibles. Considérons le cas simplifié où seules deux routes $a$ et $b$
existent entre le moniteur $m$ et la cible $t$, de longueurs respectives $|a| =
10$ et $|b| = 20$, d'une probabilité uniforme de $\frac{1}{2}$. Dès la sonde de
\ttl $10$, il suffit qu'une seule sonde emprunte la route $a$ pour que
\traceroute s'arrête; or pour observer la route $b$ dans son intégralité, il
faut au moins atteindre le \ttl 20. En particulier, la probabilité d'observer le
saut $|b| - 1$ avec \traceroute est majorée par $\frac{1}{2^{|b| - 1 - |a|}}
= \frac{1}{512}$. La probabilité d'observer le voisin de $t$ induit par $b$ est
donc très faible.

\traceroute n'est probablement pas l'outil adapté pour observer ces routes, mais
dans le cas d'\udpping, l'idée se tranpose en répétant de très nombreuses fois
l'envoi de sondes depuis un même moniteur vers une cible donnée, dans l'espoir
d'observer ces interfaces difficiles à observer. Les considérations de charge
portée par le voisinage de la cible sont à étudier, mais l'étude des routes
longues, qui sont occultées par la plupart des travaux basés sur \traceroute,
constitue selon nous une opportunité d'observer davantage d'informations, même
depuis un seul moniteur.
