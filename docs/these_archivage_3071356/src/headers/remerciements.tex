\chaptertitle*{Remerciements}
\addcontentsline{toc}{chapter}{Remerciements}

Mes premiers remerciements vont aux rapporteurs de cette thèse, Bertrand Jouve
et Jean-Jacques Pansiot, et aux examinateurs, Clémence Magnien, Pascal Mérindol
et Philippe Owezarski. Je tiens également à remercier mes encadrants de thèse,
Matthieu Latapy et Christophe Crespelle, pour votre patience à mon égard. Je
sais que je n'ai pas été un doctorant très facile à encadrer (je prépare cette
litote depuis cinq ans), et vous avez fait preuve d'une opiniâtreté, d'une
compréhension et d'une persévérance remarquables. Vous m'avez fait confiance
même quand je n'avais plus confiance moi-même, vous m'avez canalisé dans les moments
les plus chaotiques. Je suis fier d'avoir été encadré par vous. Vous allez
pouvoir reprendre une vie normale\ldots

Mes parents, Béatrice et Lionel, je ne souhaite pas me contenter de vous
remercier, je profiterai de cette saynette en marge du théâtre familial pour
déclarer que vous êtes les meilleurs parents du monde. En dépit des joutes
formelles, de la névrose pathologique que vous m'avez soigneusement inculquée,
vous m'avez également prodigué tout ce qu'un enfant peut rêver de ses parents,
et même davantage. Le goût du savoir, de la curiosité, du travail, de
l'accomplissement, ne se trouvent pas dans les gènes, mais dans l'éducation, et
c'est bien à vous deux que je la dois en premier lieu.

Je voudrais remercier deux personnes que je considère comme mes mentors. Guy
Alarcon, vous m'avez donné accès aux délices de l'abstraction et des approches
formelles, vous avez mis le feu à une poudre qui depuis, éclaire mon chemin.
Nathalie Collin, tu m'as toujours prodigué confiance, conseil et inspiration,
et ces éléments ont été et sont toujours extrêmement moteurs dans ma vie.

Je remercie mon entourage le plus proche, à commencer par Marie, mais aussi mes
frères et soeur et leurs compagnons respectifs, Benjamin et Anne-Solène, Déborah
et Guillaume, Samuel et Constance. Me suppporter au quotidien, quand j'étais
plus jeune et insupportable ou maintenant (je suis toujours insupportable),
n'est pas une mince affaire, et que nous y ayons tous survécu mérite des
féliciations et ma gratification. J'adresse également une mention particulière à
Nicolas Bonifas et Antoine Viannay, deux de mes plus vieux amis, pour leur
soutien dans les situations les plus rocambolesques.

Enfin, je souhaite remercier mes collaborateurs. Ceux du monde académique, de
l'équipe Complex Networks (ou ex-), avec qui j'ai eu la chance de travailler, ou
tout simplement discuter (que j'ai eu l'occasion d'assomer de ma cuistrerie)
lors de mes apparitions : Fabien, Jean-Lou, Lionel, Raphaël, Daniel, Maximilien,
Sergey, Alice, Sébastien, Adrien, et tous les anciens de l'équipe. Et ceux du
monde vidéo-ludique, de Millenium, qui m'ont vu mener une curieuse vie pendant
plusieurs années, en particulier mon associé Cédric Page et sa femme Naïma.

Je m'excuse de ne pas pouvoir ici remercier les nombreuses personnes qui ont eu
un impact déterminant dans ma vie, et donc dans mon travail. Si vous arrivez
jusqu'à ce paragraphe, c'est probablement que vous méritez mes remerciements
également. Bon courage pour la lecture de la suite.
